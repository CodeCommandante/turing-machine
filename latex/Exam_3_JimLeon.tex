\documentclass[12pt]{article}                                              
\documentclass[12pt]{article}                                              
\documentclass[12pt]{article}                                              
\documentclass[12pt]{article}                                              
\input{Exam_3_JimLeon.sty}
\usepackage[export]{adjustbox}
\usepackage{float}
\usepackage{wrapfig}


\title{CSC445/545 - Exam \#3 (Final)}
\author{Jim Leon}


\begin{document}
\maketitle
Due:  May 3, 2021 2pm MST.
% Pre-problem text.  The preface to the homework.  Comments which apply to the homework overall.

%%\section{} %%  This will generate a numbered problem header, but you know this now.

\section{Program Description}
%This is the problem statement.
\paragraph{}
This program explores and simulates various uses of a single-tape Turing 
Machine.  Depending on the users' selection from the various menus, the 
program will ask for a 'tape' (a user input string) and will run this tape 
through a virtual Turing Machine, simulating the moves and changes to the 
tape along the way.  The machine will prompt the user if it either completes 
the simulation (if an answer was found) or exits the simulation prematurely
(encountered a non-described transition function).

\paragraph{}
This program was written in Python version 3.8.5.  The authors machine was 
running the Linux Ubuntu 20.04.2 LTS operating system on a Intel® Core™ 
i7-6600U CPU (2.60GHz × 4) at the time of testing and release.  This 
application was not tested on any other machine and makes no guarantees to 
it's ability to run on other machines or other software configurations.  The 
author makes no warranties of support for this application after this release 
date.

\section{Example of running the application}
\paragraph{}
This is a menu-driven application.  As such, the program can be fired up by 
navigating to the applications main directory and entering the following into 
a terminal window:

\begin{verbatim}
$ ./main.py
\end{verbatim}

\subsection{Main Menu}
\paragraph{}
The program starts by displaying a brief description and author name in the 
terminal.  Directly below this there will be a Main Menu:

\begin{verbatim}
    THE TURING MACHINE SIMULATOR
==============================================================
This program simulates various uses for the Turing Machine.  
Author:  Jim Leon


------------------------- Main Menu --------------------------
1.  Language Acceptors
2.  Adders
3.  Multiply & Divide
4.  Import custom transducer from file
5.  Exit Application

Selection:  
\end{verbatim}

\paragraph{}
After the selection prompt, the user can enter the number corresponding with 
the menu choices.  This will direct them either to a sub-menu (in the cases of
menu selections 1, 2, and 3), an input command (selection 4), or will end the 
application (selection 5).  Entering numbers or other characters that do not 
correspond to the menu choices given will result in an error message and the 
reprinting of the menu:

\begin{verbatim}
Invalid selection.

------------------------- Main Menu --------------------------
1.  Language Acceptors
2.  Adders
3.  Multiply & Divide
4.  Import custom transducer from file
5.  Exit Application

Selection:     
\end{verbatim}
\paragraph{}
The application is intuitive from here.  Explanations of what constitutes 
acceptable input and what sort of constraints are instituted on that input 
should be described throughout the applications use.  Below is a brief overview 
of the other menu's the user will encounter while using the program.

\subsection{Language Acceptors}
\begin{verbatim}
-------------------- Language Acceptors ----------------------
1.  L = {(a^n)(b^n)(c^n), n > 0}
2.  L = {w w^R, |w w^R| is even, 'sigma' = (a,b)}
3.  Import custom language acceptor from file
4.  Back to Main Menu

Selection:  
\end{verbatim}

\paragraph{}
If the user selects option 1 from the Main Menu, they will be directed to the 
Language Acceptors sub-menu.  Here they have the ability to explore two 
specific simulations of a single tape Turing Machine in action by selecting 
either menu option 1 or menu option 2.  By selecting option 3, the user can 
enter the name of a custom file - \textit{the user must include the relative path of 
the file!} - from which they can run the simulation.

\paragraph{}
Selecting option 4 will return the user to the Main Menu.

\subsection{Adders}
\paragraph{}
If the user selections menu option 2 from the Main Menu, they are directed to 
the Adders sub menu:

\begin{verbatim}
-------------------------- Adders ----------------------------
1.  Add unary numbers (e.g. 111+1111)
2.  Add binary numbers (e.g. 010110+000011)
3.  Back to Main Menu

Selection:      
\end{verbatim}

\paragraph{}
Here, again, the user is given two specific simulations they can run that 
demonstrate a single-tape Turing Machine being used as a unary adder (option 
1), or a binary adder (option 2).  Menu option number 3 will return to the Main 
Menu again.

\subsection{Multiply \& Divide}
This is where things get fun!  The single-tape Turing Machine can also simulate 
multiplication and division.  By selecting menu option 3 from the Main Menu, 
the user is brought to the Multiply \& Divide sub menu:

\begin{verbatim}
----------------------- Multiply & Divide --------------------------
1.  Multiply unary numbers (e.g. 111*11 = 111111)
2.  Divide unary numbers (e.g. 111111/11 = 111)
3.  Back to Main Menu

Selection:  
\end{verbatim}

\paragraph{}
By selecting option 1, the user can watch the Turing Machine calculate unary 
multiplication; by selecting option 2, division.  Option 3 will return to the 
Main Menu.

\subsection{Import custom transducer from file}
\paragraph{}
By selecting Main Menu option 4, the user can import a file describing whatever 
custom machine they like.  The machine description file must meet certain 
conditions, however.  These are described by the application at the time of 
selection:

\begin{verbatim}
Machine description file must be of the following format:
---------------------------------------------------------
    => First n lines must describe the transition functions 
       of the form:  state, symbol -> state, symbol, moveDir
       where 'state' and 'symbol' are variables and 
       'moveDir' is either 'L' or 'R', for move Left or
       move Right, respectively.
    => 'symbol' must be a single character.
    => You must include the commas and '->' delimiter in the
       description.
    => For transitions on "blank", insert empty space between the
       commas and/or delimiters.  (See included .txt files for examples.)
    => No empty lines in the file.
    => After transitions, a line describing the starting state.
    => After starting state, a comma separated list of the 
       final states on the machine.  This is a single line.


Enter the name of the file, including it's relative path from 
this working directory:  
\end{verbatim}

\paragraph{}
One must strictly adhere to the criterion described in order for the simulation 
to work as desired.  If an incorrect path and/or file name is provided, an 
error message is displayed:

\begin{verbatim}
Could not locate file!  Please check path and/or file name and try again.
\end{verbatim}

\paragraph{}
The user is returned to the Main Menu if this error occurs.

\subsection{Exit Application}
\paragraph{}
Finally, when the user is ready to exit the application they can select option 
5.  The application closes and exclaims:

\begin{verbatim}
Goodbye!
\end{verbatim}

\section{Application Structure and Code}
\paragraph{}
The application relies on three source files: \textit{main.py}, \textit{machine.py}, 
and \textit{ui.py}.  There is also included a small test-suite in the source 
file \textit{unitTests.py} which primary tests the \textit{TuringMachine} class 
in the \textit{machine.py} file.  The other two classes, \textit{LanguageAcceptor} 
and \textit{Transducer} are basically copies of one another that abstract the 
\textit{TuringMachine} class in ways specific to their respective uses.

\subsection{main.py}
\paragraph{}
The \textit{main.py} source file contains a whopping 6 lines of code.  It 
constructs a new \textit{UI} class object (UI stands for "User Interface") 
and subsequently deletes it upon application exit:

\begin{lstlisting}
import ui

def main():
    Prog = ui.UI()
    del Prog
    return 0

main()
\end{lstlisting}

\subsection{machine.py}
\paragraph{}
This source file contains the three classes which underpin the business-logic 
of the program, \textit{TuringMachine}, \textit{LanguageAcceptor}, and \textit{
Transducer}.  As stated above the \textit{LanguageAcceptor} and \textit{
Transducer} classes are approximate copies of one another that merely abstract 
away the \textit{TuringMachine} class for their respective purposes.

\paragraph{}
Below is a listing of the public functions included in the three classes.  I've 
excluded the private helper functions and member data for the sake of brevity.

\subsubsection{TuringMachine}
\begin{lstlisting}
class TuringMachine:
    
    def getCurrentIndex(self) -> int:
        return self.__TapeIndex

    def getCurrentState(self) -> str:
        return self.__CurrState

    def getCurrentTape(self) -> list:
        return self.__Tape.copy()

    def getTapeLen(self) -> int:
        self.__trimTape()
        return len(self.__Tape)

    def isInFinalState(self) -> bool:
        for i in self.__FinalStates:
            if i == self.__CurrState:
                return True
        return False

    def move(self) -> bool:
        Trans = []
        #Check array boundary conditions
        if (self.__TapeIndex < 0) or (self.__TapeIndex >= len(self.__Tape)):
            Trans = self.__getTransOn('')
        else:
            Trans = self.__getTransOn(self.__Tape[self.__TapeIndex])
        #If returned empty array, halt state
        if len(Trans) == 0:
            return False
        #Else, perform insert/append/replace
        elif self.__TapeIndex < 0:
            self.__Tape.insert(0,Trans[3])
        elif self.__TapeIndex >= len(self.__Tape):
            self.__Tape.append(Trans[3])
        else:
            self.__Tape[self.__TapeIndex] = Trans[3]
        #Move left or right or throw exception
        if Trans[4] == 'L':
            self.__TapeIndex = self.__TapeIndex - 1
        elif Trans[4] == 'R':
            #Special case:  if 'deleting' 0 index, don't advance index.
        if Trans[3] == '' and self.__TapeIndex == 0:
            pass
        else:
            self.__TapeIndex = self.__TapeIndex + 1
        else:
            raise Exception('Machine move definition incorrect or undefined.')
        #Update current state
        self.__CurrState = Trans[2]
        self.__trimTape()
        return True
\end{lstlisting}

\subsubsection{LanguageAcceptor}
\begin{lstlisting}
class LanguageAcceptor:

    def printAlpha(self):
        print(self.__Alpha)

    def printFinalStates(self):
        print(self.__FinalStates)

    def printInitialState(self):
        print(self.__InitState)

    def printStates(self):
        print(self.__States)
    
    def printTransitions(self):
        for p in self.__TransFuncs:
            print(p)

    def run(self, Tape: str, View = True) -> bool:
        self.__TM = TuringMachine(self.__States, self.__Alpha, 
            self.__TransFuncs, self.__InitState, self.__FinalStates, 
            list(Tape))
        Running = True
        if View:
            print(Tape)
        while Running:
            Plist = self.__TM.getCurrentTape()
            Pstr = ''.join(Plist)
            if View and not DEBUG:
                print('                                  \r' + Pstr,'\r', 
                end='')
                time.sleep(0.10)
            Running = self.__TM.move()
            if DEBUG:
                print(self.__TM.getCurrentState(),' ',Pstr)
        if View and not DEBUG:
            print('\n')
        return self.__TM.isInFinalState()
\end{lstlisting}

\subsubsection{Transducer}
\begin{lstlisting}
class Transducer:

    def printAlpha(self):
        print(self.__Alpha)

    def printFinalStates(self):
        print(self.__FinalStates)

    def printInitialState(self):
        print(self.__InitState)

    def printStates(self):
        print(self.__States)
    
    def printTransitions(self):
        for p in self.__TransFuncs:
            print(p)

    def run(self, Tape: str, View = True) -> bool:
        self.__TM = TuringMachine(self.__States, self.__Alpha, 
            self.__TransFuncs, self.__InitState, self.__FinalStates, 
            list(Tape))
        Running = True
        if View:
            print(Tape)
        while Running:
            Plist = self.__TM.getCurrentTape()
            Pstr = ''.join(Plist)
            if View and not DEBUG:
                print('                                  \r' + Pstr,'\r', 
                    end='')
                time.sleep(0.10)
            Running = self.__TM.move()
            if DEBUG:
                print(self.__TM.getCurrentState(),' ',Pstr)
        if View and not DEBUG:
            print('\n')
        return self.__TM.isInFinalState()
\end{lstlisting}

\paragraph{}
If the reader wishes to explore these classes further, I encourage them to 
explore the source files.  There are some global variables that can be used 
for debugging that are not listed here.

\subsection{ui.py}
\textit{ui.py} - standing for "User Interface" - contains all of the menu and 
sub-menu functionality of the program.  After the user makes a menu selection, 
the construction of the respective classes \textit{LanguageAcceptor} or 
\textit{Transducer} are made (and subsequently deleted).  Most of the 
exception handling - in the case of a missing machine description file, for 
instance - is performed within each of the sub-menu modules/functions included 
in this source file.

\paragraph{}
Rather than exhaustively list the code for this source file here, the reader is 
encouraged to view the source file if they wish to see the implementation 
details.  Most of the sub-menu functionality is very similar.  Essentially, a 
\textit{LanguageAcceptor} or \textit{Transducer} class object is constructed, 
a tape is retrieved from user input, the simulation is run, and the class 
object is deleted and the user returned to one of the sub-menus.

\section{Program Limitations, Bugs, and To-Do's}
\paragraph{}
For all of the a-la-carte functionality in the program, exception handling and 
testing of the machine description files should ensure that the program will 
never need to be prematurely aborted.  When exceptions are caught, the user is 
typically prompted and sent back to either the Main Menu or one of the sub-menus.  
However, there are two possible scenarios wherein the program may throw an 
uncaught exception, exit unexpectedly, or need to be forceable aborted.

\subsection{Custom machine description files.}
\paragraph{}
In the case of custom machine description files (Main Menu option 4, or option 
number 3 in the Language Acceptors sub-menu), there is the possibility that 
some combination of symbols in the file may result in unexpected behavior.  In 
the typical case, where the user enters an invalid path and/or file name, the 
exception is caught, an error prompt is provided, and the program resumes 
normally.

\subsection{Dividing by zero}
\paragraph{}
For the "Divide unary numbers" module, attempting to divide by "zero" will 
put the Turing Machine into an infinite loop.  Attempting to divide by zero 
occurs when the user enters no characters after the divide ('/') character, 
like so:

\begin{verbatim}
Enter the two numbers to add, seperated by a '/' symbol (i.e. 1111/11):  11111/
\end{verbatim}

\paragraph{}
Should the user enter a string like this, they will enter an infinite loop, 
where the Turing Machine will calculate an ever-growing quotient.  If the user 
were to watch the machine run, it would look something like this:

\begin{verbatim}
11111/Q111
\end{verbatim}
.......
\begin{verbatim}
11111/Q11111111
\end{verbatim}
.......
\begin{verbatim}
11111/Q111111111111111111111111
\end{verbatim}
\paragraph{}
.......and on and on, forever.  This application loop has to be stopped 
manually be the user using a keyboard interrupt, such as CTRL-C; or 
through other forceably means.

\paragraph{}
This bug could likely be corrected with some additions to the machine 
description file \textit{unaryDivide.txt}.

\end{document}

\usepackage[export]{adjustbox}
\usepackage{float}
\usepackage{wrapfig}


\title{CSC445/545 - Exam \#3 (Final)}
\author{Jim Leon}


\begin{document}
\maketitle
Due:  May 3, 2021 2pm MST.
% Pre-problem text.  The preface to the homework.  Comments which apply to the homework overall.

%%\section{} %%  This will generate a numbered problem header, but you know this now.

\section{Program Description}
%This is the problem statement.
\paragraph{}
This program explores and simulates various uses of a single-tape Turing 
Machine.  Depending on the users' selection from the various menus, the 
program will ask for a 'tape' (a user input string) and will run this tape 
through a virtual Turing Machine, simulating the moves and changes to the 
tape along the way.  The machine will prompt the user if it either completes 
the simulation (if an answer was found) or exits the simulation prematurely
(encountered a non-described transition function).

\paragraph{}
This program was written in Python version 3.8.5.  The authors machine was 
running the Linux Ubuntu 20.04.2 LTS operating system on a Intel® Core™ 
i7-6600U CPU (2.60GHz × 4) at the time of testing and release.  This 
application was not tested on any other machine and makes no guarantees to 
it's ability to run on other machines or other software configurations.  The 
author makes no warranties of support for this application after this release 
date.

\section{Example of running the application}
\paragraph{}
This is a menu-driven application.  As such, the program can be fired up by 
navigating to the applications main directory and entering the following into 
a terminal window:

\begin{verbatim}
$ ./main.py
\end{verbatim}

\subsection{Main Menu}
\paragraph{}
The program starts by displaying a brief description and author name in the 
terminal.  Directly below this there will be a Main Menu:

\begin{verbatim}
    THE TURING MACHINE SIMULATOR
==============================================================
This program simulates various uses for the Turing Machine.  
Author:  Jim Leon


------------------------- Main Menu --------------------------
1.  Language Acceptors
2.  Adders
3.  Multiply & Divide
4.  Import custom transducer from file
5.  Exit Application

Selection:  
\end{verbatim}

\paragraph{}
After the selection prompt, the user can enter the number corresponding with 
the menu choices.  This will direct them either to a sub-menu (in the cases of
menu selections 1, 2, and 3), an input command (selection 4), or will end the 
application (selection 5).  Entering numbers or other characters that do not 
correspond to the menu choices given will result in an error message and the 
reprinting of the menu:

\begin{verbatim}
Invalid selection.

------------------------- Main Menu --------------------------
1.  Language Acceptors
2.  Adders
3.  Multiply & Divide
4.  Import custom transducer from file
5.  Exit Application

Selection:     
\end{verbatim}
\paragraph{}
The application is intuitive from here.  Explanations of what constitutes 
acceptable input and what sort of constraints are instituted on that input 
should be described throughout the applications use.  Below is a brief overview 
of the other menu's the user will encounter while using the program.

\subsection{Language Acceptors}
\begin{verbatim}
-------------------- Language Acceptors ----------------------
1.  L = {(a^n)(b^n)(c^n), n > 0}
2.  L = {w w^R, |w w^R| is even, 'sigma' = (a,b)}
3.  Import custom language acceptor from file
4.  Back to Main Menu

Selection:  
\end{verbatim}

\paragraph{}
If the user selects option 1 from the Main Menu, they will be directed to the 
Language Acceptors sub-menu.  Here they have the ability to explore two 
specific simulations of a single tape Turing Machine in action by selecting 
either menu option 1 or menu option 2.  By selecting option 3, the user can 
enter the name of a custom file - \textit{the user must include the relative path of 
the file!} - from which they can run the simulation.

\paragraph{}
Selecting option 4 will return the user to the Main Menu.

\subsection{Adders}
\paragraph{}
If the user selections menu option 2 from the Main Menu, they are directed to 
the Adders sub menu:

\begin{verbatim}
-------------------------- Adders ----------------------------
1.  Add unary numbers (e.g. 111+1111)
2.  Add binary numbers (e.g. 010110+000011)
3.  Back to Main Menu

Selection:      
\end{verbatim}

\paragraph{}
Here, again, the user is given two specific simulations they can run that 
demonstrate a single-tape Turing Machine being used as a unary adder (option 
1), or a binary adder (option 2).  Menu option number 3 will return to the Main 
Menu again.

\subsection{Multiply \& Divide}
This is where things get fun!  The single-tape Turing Machine can also simulate 
multiplication and division.  By selecting menu option 3 from the Main Menu, 
the user is brought to the Multiply \& Divide sub menu:

\begin{verbatim}
----------------------- Multiply & Divide --------------------------
1.  Multiply unary numbers (e.g. 111*11 = 111111)
2.  Divide unary numbers (e.g. 111111/11 = 111)
3.  Back to Main Menu

Selection:  
\end{verbatim}

\paragraph{}
By selecting option 1, the user can watch the Turing Machine calculate unary 
multiplication; by selecting option 2, division.  Option 3 will return to the 
Main Menu.

\subsection{Import custom transducer from file}
\paragraph{}
By selecting Main Menu option 4, the user can import a file describing whatever 
custom machine they like.  The machine description file must meet certain 
conditions, however.  These are described by the application at the time of 
selection:

\begin{verbatim}
Machine description file must be of the following format:
---------------------------------------------------------
    => First n lines must describe the transition functions 
       of the form:  state, symbol -> state, symbol, moveDir
       where 'state' and 'symbol' are variables and 
       'moveDir' is either 'L' or 'R', for move Left or
       move Right, respectively.
    => 'symbol' must be a single character.
    => You must include the commas and '->' delimiter in the
       description.
    => For transitions on "blank", insert empty space between the
       commas and/or delimiters.  (See included .txt files for examples.)
    => No empty lines in the file.
    => After transitions, a line describing the starting state.
    => After starting state, a comma separated list of the 
       final states on the machine.  This is a single line.


Enter the name of the file, including it's relative path from 
this working directory:  
\end{verbatim}

\paragraph{}
One must strictly adhere to the criterion described in order for the simulation 
to work as desired.  If an incorrect path and/or file name is provided, an 
error message is displayed:

\begin{verbatim}
Could not locate file!  Please check path and/or file name and try again.
\end{verbatim}

\paragraph{}
The user is returned to the Main Menu if this error occurs.

\subsection{Exit Application}
\paragraph{}
Finally, when the user is ready to exit the application they can select option 
5.  The application closes and exclaims:

\begin{verbatim}
Goodbye!
\end{verbatim}

\section{Application Structure and Code}
\paragraph{}
The application relies on three source files: \textit{main.py}, \textit{machine.py}, 
and \textit{ui.py}.  There is also included a small test-suite in the source 
file \textit{unitTests.py} which primary tests the \textit{TuringMachine} class 
in the \textit{machine.py} file.  The other two classes, \textit{LanguageAcceptor} 
and \textit{Transducer} are basically copies of one another that abstract the 
\textit{TuringMachine} class in ways specific to their respective uses.

\subsection{main.py}
\paragraph{}
The \textit{main.py} source file contains a whopping 6 lines of code.  It 
constructs a new \textit{UI} class object (UI stands for "User Interface") 
and subsequently deletes it upon application exit:

\begin{lstlisting}
import ui

def main():
    Prog = ui.UI()
    del Prog
    return 0

main()
\end{lstlisting}

\subsection{machine.py}
\paragraph{}
This source file contains the three classes which underpin the business-logic 
of the program, \textit{TuringMachine}, \textit{LanguageAcceptor}, and \textit{
Transducer}.  As stated above the \textit{LanguageAcceptor} and \textit{
Transducer} classes are approximate copies of one another that merely abstract 
away the \textit{TuringMachine} class for their respective purposes.

\paragraph{}
Below is a listing of the public functions included in the three classes.  I've 
excluded the private helper functions and member data for the sake of brevity.

\subsubsection{TuringMachine}
\begin{lstlisting}
class TuringMachine:
    
    def getCurrentIndex(self) -> int:
        return self.__TapeIndex

    def getCurrentState(self) -> str:
        return self.__CurrState

    def getCurrentTape(self) -> list:
        return self.__Tape.copy()

    def getTapeLen(self) -> int:
        self.__trimTape()
        return len(self.__Tape)

    def isInFinalState(self) -> bool:
        for i in self.__FinalStates:
            if i == self.__CurrState:
                return True
        return False

    def move(self) -> bool:
        Trans = []
        #Check array boundary conditions
        if (self.__TapeIndex < 0) or (self.__TapeIndex >= len(self.__Tape)):
            Trans = self.__getTransOn('')
        else:
            Trans = self.__getTransOn(self.__Tape[self.__TapeIndex])
        #If returned empty array, halt state
        if len(Trans) == 0:
            return False
        #Else, perform insert/append/replace
        elif self.__TapeIndex < 0:
            self.__Tape.insert(0,Trans[3])
        elif self.__TapeIndex >= len(self.__Tape):
            self.__Tape.append(Trans[3])
        else:
            self.__Tape[self.__TapeIndex] = Trans[3]
        #Move left or right or throw exception
        if Trans[4] == 'L':
            self.__TapeIndex = self.__TapeIndex - 1
        elif Trans[4] == 'R':
            #Special case:  if 'deleting' 0 index, don't advance index.
        if Trans[3] == '' and self.__TapeIndex == 0:
            pass
        else:
            self.__TapeIndex = self.__TapeIndex + 1
        else:
            raise Exception('Machine move definition incorrect or undefined.')
        #Update current state
        self.__CurrState = Trans[2]
        self.__trimTape()
        return True
\end{lstlisting}

\subsubsection{LanguageAcceptor}
\begin{lstlisting}
class LanguageAcceptor:

    def printAlpha(self):
        print(self.__Alpha)

    def printFinalStates(self):
        print(self.__FinalStates)

    def printInitialState(self):
        print(self.__InitState)

    def printStates(self):
        print(self.__States)
    
    def printTransitions(self):
        for p in self.__TransFuncs:
            print(p)

    def run(self, Tape: str, View = True) -> bool:
        self.__TM = TuringMachine(self.__States, self.__Alpha, 
            self.__TransFuncs, self.__InitState, self.__FinalStates, 
            list(Tape))
        Running = True
        if View:
            print(Tape)
        while Running:
            Plist = self.__TM.getCurrentTape()
            Pstr = ''.join(Plist)
            if View and not DEBUG:
                print('                                  \r' + Pstr,'\r', 
                end='')
                time.sleep(0.10)
            Running = self.__TM.move()
            if DEBUG:
                print(self.__TM.getCurrentState(),' ',Pstr)
        if View and not DEBUG:
            print('\n')
        return self.__TM.isInFinalState()
\end{lstlisting}

\subsubsection{Transducer}
\begin{lstlisting}
class Transducer:

    def printAlpha(self):
        print(self.__Alpha)

    def printFinalStates(self):
        print(self.__FinalStates)

    def printInitialState(self):
        print(self.__InitState)

    def printStates(self):
        print(self.__States)
    
    def printTransitions(self):
        for p in self.__TransFuncs:
            print(p)

    def run(self, Tape: str, View = True) -> bool:
        self.__TM = TuringMachine(self.__States, self.__Alpha, 
            self.__TransFuncs, self.__InitState, self.__FinalStates, 
            list(Tape))
        Running = True
        if View:
            print(Tape)
        while Running:
            Plist = self.__TM.getCurrentTape()
            Pstr = ''.join(Plist)
            if View and not DEBUG:
                print('                                  \r' + Pstr,'\r', 
                    end='')
                time.sleep(0.10)
            Running = self.__TM.move()
            if DEBUG:
                print(self.__TM.getCurrentState(),' ',Pstr)
        if View and not DEBUG:
            print('\n')
        return self.__TM.isInFinalState()
\end{lstlisting}

\paragraph{}
If the reader wishes to explore these classes further, I encourage them to 
explore the source files.  There are some global variables that can be used 
for debugging that are not listed here.

\subsection{ui.py}
\textit{ui.py} - standing for "User Interface" - contains all of the menu and 
sub-menu functionality of the program.  After the user makes a menu selection, 
the construction of the respective classes \textit{LanguageAcceptor} or 
\textit{Transducer} are made (and subsequently deleted).  Most of the 
exception handling - in the case of a missing machine description file, for 
instance - is performed within each of the sub-menu modules/functions included 
in this source file.

\paragraph{}
Rather than exhaustively list the code for this source file here, the reader is 
encouraged to view the source file if they wish to see the implementation 
details.  Most of the sub-menu functionality is very similar.  Essentially, a 
\textit{LanguageAcceptor} or \textit{Transducer} class object is constructed, 
a tape is retrieved from user input, the simulation is run, and the class 
object is deleted and the user returned to one of the sub-menus.

\section{Program Limitations, Bugs, and To-Do's}
\paragraph{}
For all of the a-la-carte functionality in the program, exception handling and 
testing of the machine description files should ensure that the program will 
never need to be prematurely aborted.  When exceptions are caught, the user is 
typically prompted and sent back to either the Main Menu or one of the sub-menus.  
However, there are two possible scenarios wherein the program may throw an 
uncaught exception, exit unexpectedly, or need to be forceable aborted.

\subsection{Custom machine description files.}
\paragraph{}
In the case of custom machine description files (Main Menu option 4, or option 
number 3 in the Language Acceptors sub-menu), there is the possibility that 
some combination of symbols in the file may result in unexpected behavior.  In 
the typical case, where the user enters an invalid path and/or file name, the 
exception is caught, an error prompt is provided, and the program resumes 
normally.

\subsection{Dividing by zero}
\paragraph{}
For the "Divide unary numbers" module, attempting to divide by "zero" will 
put the Turing Machine into an infinite loop.  Attempting to divide by zero 
occurs when the user enters no characters after the divide ('/') character, 
like so:

\begin{verbatim}
Enter the two numbers to add, seperated by a '/' symbol (i.e. 1111/11):  11111/
\end{verbatim}

\paragraph{}
Should the user enter a string like this, they will enter an infinite loop, 
where the Turing Machine will calculate an ever-growing quotient.  If the user 
were to watch the machine run, it would look something like this:

\begin{verbatim}
11111/Q111
\end{verbatim}
.......
\begin{verbatim}
11111/Q11111111
\end{verbatim}
.......
\begin{verbatim}
11111/Q111111111111111111111111
\end{verbatim}
\paragraph{}
.......and on and on, forever.  This application loop has to be stopped 
manually be the user using a keyboard interrupt, such as CTRL-C; or 
through other forceably means.

\paragraph{}
This bug could likely be corrected with some additions to the machine 
description file \textit{unaryDivide.txt}.

\end{document}

\usepackage[export]{adjustbox}
\usepackage{float}
\usepackage{wrapfig}


\title{CSC445/545 - Exam \#3 (Final)}
\author{Jim Leon}


\begin{document}
\maketitle
Due:  May 3, 2021 2pm MST.
% Pre-problem text.  The preface to the homework.  Comments which apply to the homework overall.

%%\section{} %%  This will generate a numbered problem header, but you know this now.

\section{Program Description}
%This is the problem statement.
\paragraph{}
This program explores and simulates various uses of a single-tape Turing 
Machine.  Depending on the users' selection from the various menus, the 
program will ask for a 'tape' (a user input string) and will run this tape 
through a virtual Turing Machine, simulating the moves and changes to the 
tape along the way.  The machine will prompt the user if it either completes 
the simulation (if an answer was found) or exits the simulation prematurely
(encountered a non-described transition function).

\paragraph{}
This program was written in Python version 3.8.5.  The authors machine was 
running the Linux Ubuntu 20.04.2 LTS operating system on a Intel® Core™ 
i7-6600U CPU (2.60GHz × 4) at the time of testing and release.  This 
application was not tested on any other machine and makes no guarantees to 
it's ability to run on other machines or other software configurations.  The 
author makes no warranties of support for this application after this release 
date.

\section{Example of running the application}
\paragraph{}
This is a menu-driven application.  As such, the program can be fired up by 
navigating to the applications main directory and entering the following into 
a terminal window:

\begin{verbatim}
$ ./main.py
\end{verbatim}

\subsection{Main Menu}
\paragraph{}
The program starts by displaying a brief description and author name in the 
terminal.  Directly below this there will be a Main Menu:

\begin{verbatim}
    THE TURING MACHINE SIMULATOR
==============================================================
This program simulates various uses for the Turing Machine.  
Author:  Jim Leon


------------------------- Main Menu --------------------------
1.  Language Acceptors
2.  Adders
3.  Multiply & Divide
4.  Import custom transducer from file
5.  Exit Application

Selection:  
\end{verbatim}

\paragraph{}
After the selection prompt, the user can enter the number corresponding with 
the menu choices.  This will direct them either to a sub-menu (in the cases of
menu selections 1, 2, and 3), an input command (selection 4), or will end the 
application (selection 5).  Entering numbers or other characters that do not 
correspond to the menu choices given will result in an error message and the 
reprinting of the menu:

\begin{verbatim}
Invalid selection.

------------------------- Main Menu --------------------------
1.  Language Acceptors
2.  Adders
3.  Multiply & Divide
4.  Import custom transducer from file
5.  Exit Application

Selection:     
\end{verbatim}
\paragraph{}
The application is intuitive from here.  Explanations of what constitutes 
acceptable input and what sort of constraints are instituted on that input 
should be described throughout the applications use.  Below is a brief overview 
of the other menu's the user will encounter while using the program.

\subsection{Language Acceptors}
\begin{verbatim}
-------------------- Language Acceptors ----------------------
1.  L = {(a^n)(b^n)(c^n), n > 0}
2.  L = {w w^R, |w w^R| is even, 'sigma' = (a,b)}
3.  Import custom language acceptor from file
4.  Back to Main Menu

Selection:  
\end{verbatim}

\paragraph{}
If the user selects option 1 from the Main Menu, they will be directed to the 
Language Acceptors sub-menu.  Here they have the ability to explore two 
specific simulations of a single tape Turing Machine in action by selecting 
either menu option 1 or menu option 2.  By selecting option 3, the user can 
enter the name of a custom file - \textit{the user must include the relative path of 
the file!} - from which they can run the simulation.

\paragraph{}
Selecting option 4 will return the user to the Main Menu.

\subsection{Adders}
\paragraph{}
If the user selections menu option 2 from the Main Menu, they are directed to 
the Adders sub menu:

\begin{verbatim}
-------------------------- Adders ----------------------------
1.  Add unary numbers (e.g. 111+1111)
2.  Add binary numbers (e.g. 010110+000011)
3.  Back to Main Menu

Selection:      
\end{verbatim}

\paragraph{}
Here, again, the user is given two specific simulations they can run that 
demonstrate a single-tape Turing Machine being used as a unary adder (option 
1), or a binary adder (option 2).  Menu option number 3 will return to the Main 
Menu again.

\subsection{Multiply \& Divide}
This is where things get fun!  The single-tape Turing Machine can also simulate 
multiplication and division.  By selecting menu option 3 from the Main Menu, 
the user is brought to the Multiply \& Divide sub menu:

\begin{verbatim}
----------------------- Multiply & Divide --------------------------
1.  Multiply unary numbers (e.g. 111*11 = 111111)
2.  Divide unary numbers (e.g. 111111/11 = 111)
3.  Back to Main Menu

Selection:  
\end{verbatim}

\paragraph{}
By selecting option 1, the user can watch the Turing Machine calculate unary 
multiplication; by selecting option 2, division.  Option 3 will return to the 
Main Menu.

\subsection{Import custom transducer from file}
\paragraph{}
By selecting Main Menu option 4, the user can import a file describing whatever 
custom machine they like.  The machine description file must meet certain 
conditions, however.  These are described by the application at the time of 
selection:

\begin{verbatim}
Machine description file must be of the following format:
---------------------------------------------------------
    => First n lines must describe the transition functions 
       of the form:  state, symbol -> state, symbol, moveDir
       where 'state' and 'symbol' are variables and 
       'moveDir' is either 'L' or 'R', for move Left or
       move Right, respectively.
    => 'symbol' must be a single character.
    => You must include the commas and '->' delimiter in the
       description.
    => For transitions on "blank", insert empty space between the
       commas and/or delimiters.  (See included .txt files for examples.)
    => No empty lines in the file.
    => After transitions, a line describing the starting state.
    => After starting state, a comma separated list of the 
       final states on the machine.  This is a single line.


Enter the name of the file, including it's relative path from 
this working directory:  
\end{verbatim}

\paragraph{}
One must strictly adhere to the criterion described in order for the simulation 
to work as desired.  If an incorrect path and/or file name is provided, an 
error message is displayed:

\begin{verbatim}
Could not locate file!  Please check path and/or file name and try again.
\end{verbatim}

\paragraph{}
The user is returned to the Main Menu if this error occurs.

\subsection{Exit Application}
\paragraph{}
Finally, when the user is ready to exit the application they can select option 
5.  The application closes and exclaims:

\begin{verbatim}
Goodbye!
\end{verbatim}

\section{Application Structure and Code}
\paragraph{}
The application relies on three source files: \textit{main.py}, \textit{machine.py}, 
and \textit{ui.py}.  There is also included a small test-suite in the source 
file \textit{unitTests.py} which primary tests the \textit{TuringMachine} class 
in the \textit{machine.py} file.  The other two classes, \textit{LanguageAcceptor} 
and \textit{Transducer} are basically copies of one another that abstract the 
\textit{TuringMachine} class in ways specific to their respective uses.

\subsection{main.py}
\paragraph{}
The \textit{main.py} source file contains a whopping 6 lines of code.  It 
constructs a new \textit{UI} class object (UI stands for "User Interface") 
and subsequently deletes it upon application exit:

\begin{lstlisting}
import ui

def main():
    Prog = ui.UI()
    del Prog
    return 0

main()
\end{lstlisting}

\subsection{machine.py}
\paragraph{}
This source file contains the three classes which underpin the business-logic 
of the program, \textit{TuringMachine}, \textit{LanguageAcceptor}, and \textit{
Transducer}.  As stated above the \textit{LanguageAcceptor} and \textit{
Transducer} classes are approximate copies of one another that merely abstract 
away the \textit{TuringMachine} class for their respective purposes.

\paragraph{}
Below is a listing of the public functions included in the three classes.  I've 
excluded the private helper functions and member data for the sake of brevity.

\subsubsection{TuringMachine}
\begin{lstlisting}
class TuringMachine:
    
    def getCurrentIndex(self) -> int:
        return self.__TapeIndex

    def getCurrentState(self) -> str:
        return self.__CurrState

    def getCurrentTape(self) -> list:
        return self.__Tape.copy()

    def getTapeLen(self) -> int:
        self.__trimTape()
        return len(self.__Tape)

    def isInFinalState(self) -> bool:
        for i in self.__FinalStates:
            if i == self.__CurrState:
                return True
        return False

    def move(self) -> bool:
        Trans = []
        #Check array boundary conditions
        if (self.__TapeIndex < 0) or (self.__TapeIndex >= len(self.__Tape)):
            Trans = self.__getTransOn('')
        else:
            Trans = self.__getTransOn(self.__Tape[self.__TapeIndex])
        #If returned empty array, halt state
        if len(Trans) == 0:
            return False
        #Else, perform insert/append/replace
        elif self.__TapeIndex < 0:
            self.__Tape.insert(0,Trans[3])
        elif self.__TapeIndex >= len(self.__Tape):
            self.__Tape.append(Trans[3])
        else:
            self.__Tape[self.__TapeIndex] = Trans[3]
        #Move left or right or throw exception
        if Trans[4] == 'L':
            self.__TapeIndex = self.__TapeIndex - 1
        elif Trans[4] == 'R':
            #Special case:  if 'deleting' 0 index, don't advance index.
        if Trans[3] == '' and self.__TapeIndex == 0:
            pass
        else:
            self.__TapeIndex = self.__TapeIndex + 1
        else:
            raise Exception('Machine move definition incorrect or undefined.')
        #Update current state
        self.__CurrState = Trans[2]
        self.__trimTape()
        return True
\end{lstlisting}

\subsubsection{LanguageAcceptor}
\begin{lstlisting}
class LanguageAcceptor:

    def printAlpha(self):
        print(self.__Alpha)

    def printFinalStates(self):
        print(self.__FinalStates)

    def printInitialState(self):
        print(self.__InitState)

    def printStates(self):
        print(self.__States)
    
    def printTransitions(self):
        for p in self.__TransFuncs:
            print(p)

    def run(self, Tape: str, View = True) -> bool:
        self.__TM = TuringMachine(self.__States, self.__Alpha, 
            self.__TransFuncs, self.__InitState, self.__FinalStates, 
            list(Tape))
        Running = True
        if View:
            print(Tape)
        while Running:
            Plist = self.__TM.getCurrentTape()
            Pstr = ''.join(Plist)
            if View and not DEBUG:
                print('                                  \r' + Pstr,'\r', 
                end='')
                time.sleep(0.10)
            Running = self.__TM.move()
            if DEBUG:
                print(self.__TM.getCurrentState(),' ',Pstr)
        if View and not DEBUG:
            print('\n')
        return self.__TM.isInFinalState()
\end{lstlisting}

\subsubsection{Transducer}
\begin{lstlisting}
class Transducer:

    def printAlpha(self):
        print(self.__Alpha)

    def printFinalStates(self):
        print(self.__FinalStates)

    def printInitialState(self):
        print(self.__InitState)

    def printStates(self):
        print(self.__States)
    
    def printTransitions(self):
        for p in self.__TransFuncs:
            print(p)

    def run(self, Tape: str, View = True) -> bool:
        self.__TM = TuringMachine(self.__States, self.__Alpha, 
            self.__TransFuncs, self.__InitState, self.__FinalStates, 
            list(Tape))
        Running = True
        if View:
            print(Tape)
        while Running:
            Plist = self.__TM.getCurrentTape()
            Pstr = ''.join(Plist)
            if View and not DEBUG:
                print('                                  \r' + Pstr,'\r', 
                    end='')
                time.sleep(0.10)
            Running = self.__TM.move()
            if DEBUG:
                print(self.__TM.getCurrentState(),' ',Pstr)
        if View and not DEBUG:
            print('\n')
        return self.__TM.isInFinalState()
\end{lstlisting}

\paragraph{}
If the reader wishes to explore these classes further, I encourage them to 
explore the source files.  There are some global variables that can be used 
for debugging that are not listed here.

\subsection{ui.py}
\textit{ui.py} - standing for "User Interface" - contains all of the menu and 
sub-menu functionality of the program.  After the user makes a menu selection, 
the construction of the respective classes \textit{LanguageAcceptor} or 
\textit{Transducer} are made (and subsequently deleted).  Most of the 
exception handling - in the case of a missing machine description file, for 
instance - is performed within each of the sub-menu modules/functions included 
in this source file.

\paragraph{}
Rather than exhaustively list the code for this source file here, the reader is 
encouraged to view the source file if they wish to see the implementation 
details.  Most of the sub-menu functionality is very similar.  Essentially, a 
\textit{LanguageAcceptor} or \textit{Transducer} class object is constructed, 
a tape is retrieved from user input, the simulation is run, and the class 
object is deleted and the user returned to one of the sub-menus.

\section{Program Limitations, Bugs, and To-Do's}
\paragraph{}
For all of the a-la-carte functionality in the program, exception handling and 
testing of the machine description files should ensure that the program will 
never need to be prematurely aborted.  When exceptions are caught, the user is 
typically prompted and sent back to either the Main Menu or one of the sub-menus.  
However, there are two possible scenarios wherein the program may throw an 
uncaught exception, exit unexpectedly, or need to be forceable aborted.

\subsection{Custom machine description files.}
\paragraph{}
In the case of custom machine description files (Main Menu option 4, or option 
number 3 in the Language Acceptors sub-menu), there is the possibility that 
some combination of symbols in the file may result in unexpected behavior.  In 
the typical case, where the user enters an invalid path and/or file name, the 
exception is caught, an error prompt is provided, and the program resumes 
normally.

\subsection{Dividing by zero}
\paragraph{}
For the "Divide unary numbers" module, attempting to divide by "zero" will 
put the Turing Machine into an infinite loop.  Attempting to divide by zero 
occurs when the user enters no characters after the divide ('/') character, 
like so:

\begin{verbatim}
Enter the two numbers to add, seperated by a '/' symbol (i.e. 1111/11):  11111/
\end{verbatim}

\paragraph{}
Should the user enter a string like this, they will enter an infinite loop, 
where the Turing Machine will calculate an ever-growing quotient.  If the user 
were to watch the machine run, it would look something like this:

\begin{verbatim}
11111/Q111
\end{verbatim}
.......
\begin{verbatim}
11111/Q11111111
\end{verbatim}
.......
\begin{verbatim}
11111/Q111111111111111111111111
\end{verbatim}
\paragraph{}
.......and on and on, forever.  This application loop has to be stopped 
manually be the user using a keyboard interrupt, such as CTRL-C; or 
through other forceably means.

\paragraph{}
This bug could likely be corrected with some additions to the machine 
description file \textit{unaryDivide.txt}.

\end{document}

\usepackage[export]{adjustbox}
\usepackage{float}
\usepackage{wrapfig}


\title{CSC445/545 - Exam \#3 (Final)}
\author{Jim Leon}


\begin{document}
\maketitle
Due:  May 3, 2021 2pm MST.
% Pre-problem text.  The preface to the homework.  Comments which apply to the homework overall.

%%\section{} %%  This will generate a numbered problem header, but you know this now.

\section{Program Description}
%This is the problem statement.
\paragraph{}
This program explores and simulates various uses of a single-tape Turing 
Machine.  Depending on the users' selection from the various menus, the 
program will ask for a 'tape' (a user input string) and will run this tape 
through a virtual Turing Machine, simulating the moves and changes to the 
tape along the way.  The machine will prompt the user if it either completes 
the simulation (if an answer was found) or exits the simulation prematurely
(encountered a non-described transition function).

\paragraph{}
This program was written in Python version 3.8.5.  The authors machine was 
running the Linux Ubuntu 20.04.2 LTS operating system on a Intel® Core™ 
i7-6600U CPU (2.60GHz × 4) at the time of testing and release.  This 
application was not tested on any other machine and makes no guarantees to 
it's ability to run on other machines or other software configurations.  The 
author makes no warranties of support for this application after this release 
date.

\section{Example of running the application}
\paragraph{}
This is a menu-driven application.  As such, the program can be fired up by 
navigating to the applications main directory and entering the following into 
a terminal window:

\begin{verbatim}
$ ./main.py
\end{verbatim}

\subsection{Main Menu}
\paragraph{}
The program starts by displaying a brief description and author name in the 
terminal.  Directly below this there will be a Main Menu:

\begin{verbatim}
    THE TURING MACHINE SIMULATOR
==============================================================
This program simulates various uses for the Turing Machine.  
Author:  Jim Leon


------------------------- Main Menu --------------------------
1.  Language Acceptors
2.  Adders
3.  Multiply & Divide
4.  Import custom transducer from file
5.  Exit Application

Selection:  
\end{verbatim}

\paragraph{}
After the selection prompt, the user can enter the number corresponding with 
the menu choices.  This will direct them either to a sub-menu (in the cases of
menu selections 1, 2, and 3), an input command (selection 4), or will end the 
application (selection 5).  Entering numbers or other characters that do not 
correspond to the menu choices given will result in an error message and the 
reprinting of the menu:

\begin{verbatim}
Invalid selection.

------------------------- Main Menu --------------------------
1.  Language Acceptors
2.  Adders
3.  Multiply & Divide
4.  Import custom transducer from file
5.  Exit Application

Selection:     
\end{verbatim}
\paragraph{}
The application is intuitive from here.  Explanations of what constitutes 
acceptable input and what sort of constraints are instituted on that input 
should be described throughout the applications use.  Below is a brief overview 
of the other menu's the user will encounter while using the program.

\subsection{Language Acceptors}
\begin{verbatim}
-------------------- Language Acceptors ----------------------
1.  L = {(a^n)(b^n)(c^n), n > 0}
2.  L = {w w^R, |w w^R| is even, 'sigma' = (a,b)}
3.  Import custom language acceptor from file
4.  Back to Main Menu

Selection:  
\end{verbatim}

\paragraph{}
If the user selects option 1 from the Main Menu, they will be directed to the 
Language Acceptors sub-menu.  Here they have the ability to explore two 
specific simulations of a single tape Turing Machine in action by selecting 
either menu option 1 or menu option 2.  By selecting option 3, the user can 
enter the name of a custom file - \textit{the user must include the relative path of 
the file!} - from which they can run the simulation.

\paragraph{}
Selecting option 4 will return the user to the Main Menu.

\subsection{Adders}
\paragraph{}
If the user selections menu option 2 from the Main Menu, they are directed to 
the Adders sub menu:

\begin{verbatim}
-------------------------- Adders ----------------------------
1.  Add unary numbers (e.g. 111+1111)
2.  Add binary numbers (e.g. 010110+000011)
3.  Back to Main Menu

Selection:      
\end{verbatim}

\paragraph{}
Here, again, the user is given two specific simulations they can run that 
demonstrate a single-tape Turing Machine being used as a unary adder (option 
1), or a binary adder (option 2).  Menu option number 3 will return to the Main 
Menu again.

\subsection{Multiply \& Divide}
This is where things get fun!  The single-tape Turing Machine can also simulate 
multiplication and division.  By selecting menu option 3 from the Main Menu, 
the user is brought to the Multiply \& Divide sub menu:

\begin{verbatim}
----------------------- Multiply & Divide --------------------------
1.  Multiply unary numbers (e.g. 111*11 = 111111)
2.  Divide unary numbers (e.g. 111111/11 = 111)
3.  Back to Main Menu

Selection:  
\end{verbatim}

\paragraph{}
By selecting option 1, the user can watch the Turing Machine calculate unary 
multiplication; by selecting option 2, division.  Option 3 will return to the 
Main Menu.

\subsection{Import custom transducer from file}
\paragraph{}
By selecting Main Menu option 4, the user can import a file describing whatever 
custom machine they like.  The machine description file must meet certain 
conditions, however.  These are described by the application at the time of 
selection:

\begin{verbatim}
Machine description file must be of the following format:
---------------------------------------------------------
    => First n lines must describe the transition functions 
       of the form:  state, symbol -> state, symbol, moveDir
       where 'state' and 'symbol' are variables and 
       'moveDir' is either 'L' or 'R', for move Left or
       move Right, respectively.
    => 'symbol' must be a single character.
    => You must include the commas and '->' delimiter in the
       description.
    => For transitions on "blank", insert empty space between the
       commas and/or delimiters.  (See included .txt files for examples.)
    => No empty lines in the file.
    => After transitions, a line describing the starting state.
    => After starting state, a comma separated list of the 
       final states on the machine.  This is a single line.


Enter the name of the file, including it's relative path from 
this working directory:  
\end{verbatim}

\paragraph{}
One must strictly adhere to the criterion described in order for the simulation 
to work as desired.  If an incorrect path and/or file name is provided, an 
error message is displayed:

\begin{verbatim}
Could not locate file!  Please check path and/or file name and try again.
\end{verbatim}

\paragraph{}
The user is returned to the Main Menu if this error occurs.

\subsection{Exit Application}
\paragraph{}
Finally, when the user is ready to exit the application they can select option 
5.  The application closes and exclaims:

\begin{verbatim}
Goodbye!
\end{verbatim}

\section{Application Structure and Code}
\paragraph{}
The application relies on three source files: \textit{main.py}, \textit{machine.py}, 
and \textit{ui.py}.  There is also included a small test-suite in the source 
file \textit{unitTests.py} which primary tests the \textit{TuringMachine} class 
in the \textit{machine.py} file.  The other two classes, \textit{LanguageAcceptor} 
and \textit{Transducer} are basically copies of one another that abstract the 
\textit{TuringMachine} class in ways specific to their respective uses.

\subsection{main.py}
\paragraph{}
The \textit{main.py} source file contains a whopping 6 lines of code.  It 
constructs a new \textit{UI} class object (UI stands for "User Interface") 
and subsequently deletes it upon application exit:

\begin{lstlisting}
import ui

def main():
    Prog = ui.UI()
    del Prog
    return 0

main()
\end{lstlisting}

\subsection{machine.py}
\paragraph{}
This source file contains the three classes which underpin the business-logic 
of the program, \textit{TuringMachine}, \textit{LanguageAcceptor}, and \textit{
Transducer}.  As stated above the \textit{LanguageAcceptor} and \textit{
Transducer} classes are approximate copies of one another that merely abstract 
away the \textit{TuringMachine} class for their respective purposes.

\paragraph{}
Below is a listing of the public functions included in the three classes.  I've 
excluded the private helper functions and member data for the sake of brevity.

\subsubsection{TuringMachine}
\begin{lstlisting}
class TuringMachine:
    
    def getCurrentIndex(self) -> int:
        return self.__TapeIndex

    def getCurrentState(self) -> str:
        return self.__CurrState

    def getCurrentTape(self) -> list:
        return self.__Tape.copy()

    def getTapeLen(self) -> int:
        self.__trimTape()
        return len(self.__Tape)

    def isInFinalState(self) -> bool:
        for i in self.__FinalStates:
            if i == self.__CurrState:
                return True
        return False

    def move(self) -> bool:
        Trans = []
        #Check array boundary conditions
        if (self.__TapeIndex < 0) or (self.__TapeIndex >= len(self.__Tape)):
            Trans = self.__getTransOn('')
        else:
            Trans = self.__getTransOn(self.__Tape[self.__TapeIndex])
        #If returned empty array, halt state
        if len(Trans) == 0:
            return False
        #Else, perform insert/append/replace
        elif self.__TapeIndex < 0:
            self.__Tape.insert(0,Trans[3])
        elif self.__TapeIndex >= len(self.__Tape):
            self.__Tape.append(Trans[3])
        else:
            self.__Tape[self.__TapeIndex] = Trans[3]
        #Move left or right or throw exception
        if Trans[4] == 'L':
            self.__TapeIndex = self.__TapeIndex - 1
        elif Trans[4] == 'R':
            #Special case:  if 'deleting' 0 index, don't advance index.
        if Trans[3] == '' and self.__TapeIndex == 0:
            pass
        else:
            self.__TapeIndex = self.__TapeIndex + 1
        else:
            raise Exception('Machine move definition incorrect or undefined.')
        #Update current state
        self.__CurrState = Trans[2]
        self.__trimTape()
        return True
\end{lstlisting}

\subsubsection{LanguageAcceptor}
\begin{lstlisting}
class LanguageAcceptor:

    def printAlpha(self):
        print(self.__Alpha)

    def printFinalStates(self):
        print(self.__FinalStates)

    def printInitialState(self):
        print(self.__InitState)

    def printStates(self):
        print(self.__States)
    
    def printTransitions(self):
        for p in self.__TransFuncs:
            print(p)

    def run(self, Tape: str, View = True) -> bool:
        self.__TM = TuringMachine(self.__States, self.__Alpha, 
            self.__TransFuncs, self.__InitState, self.__FinalStates, 
            list(Tape))
        Running = True
        if View:
            print(Tape)
        while Running:
            Plist = self.__TM.getCurrentTape()
            Pstr = ''.join(Plist)
            if View and not DEBUG:
                print('                                  \r' + Pstr,'\r', 
                end='')
                time.sleep(0.10)
            Running = self.__TM.move()
            if DEBUG:
                print(self.__TM.getCurrentState(),' ',Pstr)
        if View and not DEBUG:
            print('\n')
        return self.__TM.isInFinalState()
\end{lstlisting}

\subsubsection{Transducer}
\begin{lstlisting}
class Transducer:

    def printAlpha(self):
        print(self.__Alpha)

    def printFinalStates(self):
        print(self.__FinalStates)

    def printInitialState(self):
        print(self.__InitState)

    def printStates(self):
        print(self.__States)
    
    def printTransitions(self):
        for p in self.__TransFuncs:
            print(p)

    def run(self, Tape: str, View = True) -> bool:
        self.__TM = TuringMachine(self.__States, self.__Alpha, 
            self.__TransFuncs, self.__InitState, self.__FinalStates, 
            list(Tape))
        Running = True
        if View:
            print(Tape)
        while Running:
            Plist = self.__TM.getCurrentTape()
            Pstr = ''.join(Plist)
            if View and not DEBUG:
                print('                                  \r' + Pstr,'\r', 
                    end='')
                time.sleep(0.10)
            Running = self.__TM.move()
            if DEBUG:
                print(self.__TM.getCurrentState(),' ',Pstr)
        if View and not DEBUG:
            print('\n')
        return self.__TM.isInFinalState()
\end{lstlisting}

\paragraph{}
If the reader wishes to explore these classes further, I encourage them to 
explore the source files.  There are some global variables that can be used 
for debugging that are not listed here.

\subsection{ui.py}
\textit{ui.py} - standing for "User Interface" - contains all of the menu and 
sub-menu functionality of the program.  After the user makes a menu selection, 
the construction of the respective classes \textit{LanguageAcceptor} or 
\textit{Transducer} are made (and subsequently deleted).  Most of the 
exception handling - in the case of a missing machine description file, for 
instance - is performed within each of the sub-menu modules/functions included 
in this source file.

\paragraph{}
Rather than exhaustively list the code for this source file here, the reader is 
encouraged to view the source file if they wish to see the implementation 
details.  Most of the sub-menu functionality is very similar.  Essentially, a 
\textit{LanguageAcceptor} or \textit{Transducer} class object is constructed, 
a tape is retrieved from user input, the simulation is run, and the class 
object is deleted and the user returned to one of the sub-menus.

\section{Program Limitations, Bugs, and To-Do's}
\paragraph{}
For all of the a-la-carte functionality in the program, exception handling and 
testing of the machine description files should ensure that the program will 
never need to be prematurely aborted.  When exceptions are caught, the user is 
typically prompted and sent back to either the Main Menu or one of the sub-menus.  
However, there are two possible scenarios wherein the program may throw an 
uncaught exception, exit unexpectedly, or need to be forceable aborted.

\subsection{Custom machine description files.}
\paragraph{}
In the case of custom machine description files (Main Menu option 4, or option 
number 3 in the Language Acceptors sub-menu), there is the possibility that 
some combination of symbols in the file may result in unexpected behavior.  In 
the typical case, where the user enters an invalid path and/or file name, the 
exception is caught, an error prompt is provided, and the program resumes 
normally.

\subsection{Dividing by zero}
\paragraph{}
For the "Divide unary numbers" module, attempting to divide by "zero" will 
put the Turing Machine into an infinite loop.  Attempting to divide by zero 
occurs when the user enters no characters after the divide ('/') character, 
like so:

\begin{verbatim}
Enter the two numbers to add, seperated by a '/' symbol (i.e. 1111/11):  11111/
\end{verbatim}

\paragraph{}
Should the user enter a string like this, they will enter an infinite loop, 
where the Turing Machine will calculate an ever-growing quotient.  If the user 
were to watch the machine run, it would look something like this:

\begin{verbatim}
11111/Q111
\end{verbatim}
.......
\begin{verbatim}
11111/Q11111111
\end{verbatim}
.......
\begin{verbatim}
11111/Q111111111111111111111111
\end{verbatim}
\paragraph{}
.......and on and on, forever.  This application loop has to be stopped 
manually be the user using a keyboard interrupt, such as CTRL-C; or 
through other forceably means.

\paragraph{}
This bug could likely be corrected with some additions to the machine 
description file \textit{unaryDivide.txt}.

\end{document}
